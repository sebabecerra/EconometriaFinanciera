\documentclass{beamer}
\usepackage[document]{ragged2e}
\usepackage[spanish]{babel}
\usepackage[utf8]{inputenc}
\usepackage{lmodern}
\usepackage{amsmath}
\mode<presentation> {
	%\usetheme{default} % liso sin nada bueno simple
	%\usetheme{Pittsburgh} % liso sin nada bueno simple
	%\usetheme{Montpellier} % liso con orden arriba
	%\usetheme{Singapore} % orden arriba liso (mas oscuro)
	%\usetheme{Boadilla} % parecido al liso, con pie de pagina
	
	%\usetheme{Luebeck} % encabezado y pie de pagina (oscuro)
	%\usetheme{Copenhagen} % encabezado y pie de pagina
	\usetheme{Madrid} % encabezado y pie de pagina (azul)
		
	%\usetheme{Goettingen} % con orden a la derecha
	%\usetheme{Marburg} % con orden a la derecha (Oscuro)
	%\usetheme{Hannover} % con orden a la izquierda
	
	%%%%% Colores %%%%%
	%\usecolortheme{albatross} % fondo azul
	%\usecolortheme{beetle} % fondo plomo
		
	%\usecolortheme{dove} % quita todo el color, blanco
	%\usecolortheme{fly} % todo plomo

	%\usecolortheme{crane} % encabezado pie de pagina, naranjo
	%\usecolortheme{seagull} % encabezado pie de pagina, plomo
	\usecolortheme{seahorse} % encabezado pie de pagina azulplomo
	%\usecolortheme{wolverine} % encabezado pie de pagina, amarillo
}
\usepackage{graphicx}
\usepackage{subfigure}
%----------------------------------------------------------------------------------------
%	TITLE PAGE
%----------------------------------------------------------------------------------------

\title[Econometr\'ia Financiera]{Econometr\'ia Financiera}
\subtitle{2022\\
Tema 4: Metodolog\'ia de Box Jenkins (ARIMA)}
\author[Rodrigo Ortiz] {Rodrigo Ortiz}
\institute[UAH]{
	Universidad Adolfo Ib\'a\~nez \\
}
\date{Chile, 2022}
%---------------------------------------------------------
% Presentaci�n
%---------------------------------------------------------
\begin{document}
\begin{frame}
	\titlepage
\end{frame}
%---------------------------------------------------------
% Indice
%---------------------------------------------------------
\begin{frame}
	\frametitle{Planificaci\'on}
	\framesubtitle{Agenda}
	\tableofcontents
\end{frame}

\section{Series de tiempo no estacionarias}
%%%%%%%%%%%%%%%%%%%%%%%%%%%%
%%%%%%%%%%%%%%%%%%%%%%%%%%%%
\begin{frame}
\frametitle{Series de tiempo}
\framesubtitle{No estacionarias}

Hasta ahora nos hemos enfocado en series de tipo \textbf{retorno} que son estacionarias (o al menos razonablemnte).

\vspace{0.5cm}

Sin embargo, algunas series de precios de activos financieros o tipos de cambio tienen a ser \textbf{no estacionarios}.

\vspace{0.5cm}

Muchas veces, en series econ\'omicas y/o financieras, esta \textbf{no estacionariedad} es generada por la presencia de ra\'ices unitarias.
\end{frame}


%%%%%%%%%%%%%%%%%%%%%%%%%%%%
%%%%%%%%%%%%%%%%%%%%%%%%%%%%
\begin{frame}
\frametitle{Series de tiempo}
\framesubtitle{No estacionarias:  Random Walk}

Un ejemplo cl\'asico de \textbf{serie no estacionaria con ra\'iz unitaria} es el paseo aleatorio (randon walk)

\vspace{0.5cm}

Sea $\lbrace P_t \rbrace$ una serie de tiempo, diremos que $\lbrace P_t \rbrace$  es un RW si satisface:

\begin{align*}
P_t &= P_{t-1}+ \epsilon_t
\end{align*}


Si pensamoq $\lbrace P_t \rbrace$ como un caso particular de un AR(1) entonces el coeficiente autoregresivo $\rho = 1$, lo cual \textbf{no satisface las condiciones de estacionariedad}.

\end{frame}


%%%%%%%%%%%%%%%%%%%%%%%%%%%%
%%%%%%%%%%%%%%%%%%%%%%%%%%%%
\begin{frame}
\frametitle{Series de tiempo}
\framesubtitle{ARIMA(p,d,q)}

Consideremos un modelo ARMA(p,q)

\begin{align*}
Y_t = c + \rho_1 Y_{t-1}+\dots+\rho_p Y_{t-p}+ & \epsilon_t - \theta_1 \epsilon_{t-1}-\dots - \theta_q \epsilon_{t-q}\\
(1-\rho_1 L-\dots-\rho_pL^p)Y_t & = c+(1-\theta_1 L-\dots-\theta_pL^p)\epsilon_t \\
\rho(L)Y_t & = c + \theta(L)\epsilon_t
\end{align*}

Si un modelo ARMA tiene una ra\'iz unitaria, entonces el modelo se vuelve un \textbf{ARIMA(p,d,q)}.

\vspace{0.5cm}

As\'i, si un modelo \textbf{ARIMA} posee una ra\'iz unitaria, es porque el polinomio \textbf{AR} posee una.

\end{frame}

%%%%%%%%%%%%%%%%%%%%%%%%%%%%
%%%%%%%%%%%%%%%%%%%%%%%%%%%%
\begin{frame}
\frametitle{Series de tiempo}
\framesubtitle{ARIMA(p,d,q)}

El enfoque cl\'asico es \textbf{diferenciar}. Una serie $Y_t$ sigue un \textbf{ARIMA(p,d,q)} si

\begin{align*}
W_t &= Y_t - Y_{t-1} = (1-L) Y_t
\end{align*}

sigue un proceso estacionario \textbf{ARMA(p,q)}

\end{frame}


%%%%%%%%%%%%%%%%%%%%%%%%%%%%
%%%%%%%%%%%%%%%%%%%%%%%%%%%%
\begin{frame}
\frametitle{Series de tiempo}
\framesubtitle{ARIMA(p,d,q)}

La idea de transformar una serie \textbf{no estacionaria} (por ejemplo el log del precio) en una \textbf{estacionaria} (retornos) considerando sus \textit{cambios} se denomina \textbf{diferenciar}.

\vspace{0.5cm}

$W_t = Y_t - Y_{t-1} = \Delta Y_t$ es la primera diferencia de $Y_t$

\vspace{0.5cm}

Podr\'ia ser el caso que $Y_t$ contenga m\'ultiples ra\'ices unitarias y por tanto, sea necesario diferenciar m\'as de una vez para ser estacionaria.

\end{frame}

%%%%%%%%%%%%%%%%%%%%%%%%%%%%
%%%%%%%%%%%%%%%%%%%%%%%%%%%%
\begin{frame}
\frametitle{Series de tiempo}
\framesubtitle{ARIMA(p,d,q)}

\textbf{Por ejemplo}: Sea $Y_t$ no estacionario, entonces calculamos la primera diferencia como $W_t = Y_t - Y_{t-1}$, si esta primera diferencia es no estacionaria, entonces calculamos la segunda diferencia como 

\vspace{0.5cm}

$Z_t = W_t - W_{t-1} = Y_t - Y_{t-1} - Y_{t-1} + Y_{t-2}$

\vspace{0.3cm}

$ = Y_t -2 Y_{t-1} + Y_{t-2} =(1-2L+L^2)Y_t = (1-L)^2Y_t$


\vspace{0.5cm}
 As\'i, si $Z_t$ estacionaria entonces $Z_t \sim ARMA(p,q)$ y por consiguiente $Y_t \sim ARIMA(p,2,q)$
\end{frame}

%%%%%%%%%%%%%%%%%%%%%%%%%%%%
%%%%%%%%%%%%%%%%%%%%%%%%%%%%
\begin{frame}
\frametitle{Series de tiempo}
\framesubtitle{ARIMA(p,d,q)}

Como notaci\'on general para $Y_t \sim ARIMA(p,d,q)$

\begin{align*}
\rho (L) (1-L)^d Y_t & = c + \theta (L) \epsilon_t
\end{align*}

\textbf{Nota}: Estamos permitiendo que p,d,q $\in \mathbb{N}$, existen otro tipo de modelos los ARFIMA que permiten \textit{d} \textbf{racional}.

\end{frame}


%%%%%%%%%%%%%%%%%%%%%%%%%%%%
%%%%%%%%%%%%%%%%%%%%%%%%%%%%
\begin{frame}
\frametitle{Series de tiempo}
\framesubtitle{Ejemplo primera diferencia}

Sea el siguiente proceso $Y_t = \alpha t + \epsilon_t$

\vspace{0.6cm}

\begin{enumerate}
\item[1.] Demuestre que $Y_t$ es no estacionario.
\item[2.] Calcule la primera diferencia del proceso $Y_t$.
\item[3.] Demuestre que la primera diferencia de $Y_t$ es un proceso estacionario e indique el modelo que describe al proceso $Y_t$.
\end{enumerate}
\end{frame}


%%%%%%%%%%%%%%%%%%%%%%%%%%%%
%%%%%%%%%%%%%%%%%%%%%%%%%%%%
\begin{frame}
\frametitle{Series de tiempo}
\framesubtitle{Ejemplo primera diferencia: soluci\'on}

Sea el siguiente proceso $Y_t = \alpha t + \epsilon_t$

\vspace{0.6cm}

\begin{enumerate}
\item[1.] Demuestre que $Y_t$ es no estacionario.

\begin{align*}
E[Y_t] & =\alpha t 
\end{align*}

Por lo tanto, es no estacionario.
\end{enumerate}

\end{frame}


\begin{frame}
\frametitle{Series de tiempo}
\framesubtitle{Ejemplo primera diferencia: soluci\'on}

\begin{enumerate}
\item[2]. Calcule la primera diferencia del proceso $Y_t$.

\vspace{0.5cm}

Sea $W_t = \Delta Y_t = Y_t-Y_{t-1} = (1-L)^{d=1}Y_t$

\vspace{0.5cm}

$= \alpha t + \epsilon_t -\alpha(t-1)-\epsilon_{t-1} = \alpha + \epsilon_t - \epsilon_{t-1}$
\vspace{0.5cm}
\item[3.] Demuestre que la primera diferencia de $Y_t$ es un proceso estacionario e indique el modelo que describe al proceso $Y_t$.

\vspace{0.5cm}

Demostraci\'on queda de TAREA.

\vspace{0.5cm}

As\'i, $W_t$ es estacionario, $W_t \sim ARMA(0,1)$. 
\vspace{0.5cm}
Por lo tanto, $Y_t \sim ARIMA(0,1,1)$
\end{enumerate}
\end{frame}


%%%%%%%%%%%%%%%%%%%%%%%%%%%%
%%%%%%%%%%%%%%%%%%%%%%%%%%%%
\begin{frame}
\frametitle{Series de tiempo}
\framesubtitle{Ejemplo segunda diferencia}

Sea el siguiente proceso $Y_t = \alpha t^2 + \epsilon_t$

\vspace{0.6cm}

\begin{enumerate}
\item[1.] Demuestre que $Y_t$ es no estacionario.
\item[2.] Calcule la primera diferencia del proceso $Y_t$ y demuestre que no es estacionaria.
\item[3.] Demuestre que la segunda diferencia de $Y_t$ es un proceso estacionario e indique el modelo que describe al proceso $Y_t$.
\end{enumerate}
\end{frame}



%%%%%%%%%%%%%%%%%%%%%%%%%%%%
%%%%%%%%%%%%%%%%%%%%%%%%%%%%
\begin{frame}
\frametitle{Series de tiempo}
\framesubtitle{Ejemplo segunda diferencia: soluci\'on}

Sea el siguiente proceso $Y_t = \alpha t^2 + \epsilon_t$

\vspace{0.6cm}

\begin{enumerate}
\item[1.] Demuestre que $Y_t$ es no estacionario

\begin{align*}
E[Y_t] & =\alpha t^2
\end{align*}

Por lo tanto, es no estacionario.
\end{enumerate}
\end{frame}




%%%%%%%%%%%%%%%%%%%%%%%%%%%%
%%%%%%%%%%%%%%%%%%%%%%%%%%%%
\begin{frame}
\frametitle{Series de tiempo}
\framesubtitle{Ejemplo segunda diferencia: soluci\'on}

\begin{enumerate}
\item[2.] Calcule la primera diferencia del proceso $Y_t$ y demuestre que no es estacionaria.

\vspace{0.5cm}

Sea $W_t = \Delta Y_t = Y_t-Y_{t-1} = (1-L)^{d=1}Y_t$

\vspace{0.5cm}

$= \alpha t^2 + \epsilon_t -\alpha(t-1)^2-\epsilon_{t-1} = 2\alpha t - \alpha + \epsilon_t - \epsilon_{t-1}$

\begin{align*}
E[W_t] & =2\alpha t - \alpha
\end{align*}

Por lo tanto, $W_t$ es no estacionario.

\end{enumerate}
\end{frame}


%%%%%%%%%%%%%%%%%%%%%%%%%%%%
%%%%%%%%%%%%%%%%%%%%%%%%%%%%
\begin{frame}
\frametitle{Series de tiempo}
\framesubtitle{Ejemplo segunda diferencia: soluci\'on}

\begin{enumerate}
\item[3.] Demuestre que la segunda diferencia de $Y_t$ es un proceso estacionario e indique el modelo que describe al proceso $Y_t$.

\vspace{0.3cm}
Sea $Z_t = W_t - W_{t-1}$
\vspace{0.3cm}

$= 2\alpha t - \alpha + \epsilon_t - \epsilon_{t-1} - 2\alpha (t-1) - \alpha - \epsilon_{t-1} + \epsilon_{t-2}$

\vspace{0.3cm}

$= 2\alpha + \epsilon_t- 2\epsilon_{t-1} + \epsilon_{t-2}$
\vspace{0.3cm}

Demostraci\'on queda de TAREA.

\vspace{0.5cm}

As\'i, $Z_t$ es estacionario, $Z_t \sim ARMA(0,2)$. 
\vspace{0.5cm}
Por lo tanto, $Y_t \sim ARIMA(0,2,2)$
\end{enumerate}
\end{frame}


%%%%%%%%%%%%%%%%%%%%%%%%%%%%
%%%%%%%%%%%%%%%%%%%%%%%%%%%%
\begin{frame}
\frametitle{Series de tiempo}
\framesubtitle{No estacionaria}

\textbf{Definici\'on}: Sea $Y_t$ una serie de tiempo \textbf{no} estacionaria, diremos que $Y_t$ es integrable de orden \textbf{d} si y s\'olo si

\begin{align*}
W_t & = (1-L)^d Y_t \sim estacionario
\end{align*}
\end{frame}


\section{Pron\'osticos ARIMA}

%%%%%%%%%%%%%%%%%%%%%%%%%%%%
%%%%%%%%%%%%%%%%%%%%%%%%%%%%
\begin{frame}
\frametitle{Series de tiempo}
\framesubtitle{Pron\'osticos ARIMA: ejemplo}

Para el c\'alculo de errores, los errores se pronostican como cero sobre el set de informaci\'on.

\vspace{0.5cm}

\textbf{Ejemplo}: Sea el proceso $Y_{t+1} = c + \rho Y_t + \epsilon_{t+1}$ con $|\rho| < 1$

\vspace{0.5cm}

\begin{enumerate}
\item[1.] Calcule el pron\'ostico de $Y_t$ un paso hacia adelante.
\item[2.] Calcule el error de pron\'ostico de $Y_t$ un paso hacia adelante.
\item[3.] Calcule el pron\'ostico de $Y_t$ dos pasos hacia adelante.
\item[4.] Calcule el error de pron\'ostico de $Y_t$ dos pasos hacia adelante.
\item[5.] Obtenga una expresi\'on general para el pron\'osticos \textbf{h} pasos hacia adelante.
\item[6.] Calcule el error cuadr\'atico medio de estimaci\'on \textbf{h} pasos hacia adelante.
\end{enumerate}
\end{frame}


%%%%%%%%%%%%%%%%%%%%%%%%%%%%
%%%%%%%%%%%%%%%%%%%%%%%%%%%%
\begin{frame}
\frametitle{Series de tiempo}
\framesubtitle{Pron\'osticos ARIMA: soluci\'on}


\textbf{Ejemplo}: Sea el proceso $Y_{t+1} = c + \rho Y_t + \epsilon_{t+1}$

\vspace{0.5cm}

\begin{enumerate}
\item[1.] Calcule el pron\'ostico de $Y_t$ un paso hacia adelante.

\begin{align*}
Y_t^F(1) & = c + \rho Y_t
\end{align*}

\item[2.] Calcule el error de pron\'ostico de $Y_t$ un paso hacia adelante.

\begin{align*}
e_t^F(1) & = Y_{t+1} - Y_t^F(1) =  \epsilon_{t+1}
\end{align*}
\end{enumerate}
\end{frame}


%%%%%%%%%%%%%%%%%%%%%%%%%%%%
%%%%%%%%%%%%%%%%%%%%%%%%%%%%
\begin{frame}
\frametitle{Series de tiempo}
\framesubtitle{Pron\'osticos ARIMA: ejemplo}

\textbf{Ejemplo}: Sea el proceso $Y_{t+1} = c + \rho Y_t + \epsilon_{t+1}$

\vspace{0.3cm}

\begin{enumerate}
\item[3.] Calcule el pron\'ostico de $Y_t$ dos pasos hacia adelante.

\vspace{-0.8cm}

\begin{align*}
Y_t^F(2) & = c + \rho Y_t^F(1) = c + \rho [c+\rho Y_t] \\
& = c(1+\rho) + \rho^2 Y_t
\end{align*}

\item[4.] Calcule el error de pron\'ostico de $Y_t$ dos pasos hacia adelante.

\vspace{-0.8cm}

\begin{align*}
Y_{t+2} & = c + \rho Y_{t+1} + \epsilon_{t+2} \\
& = c + \rho [c + \rho Y_t + \epsilon_{t+1}] + \epsilon_{t+2}\\
& = c(1+\rho) + \rho^2 Y_t + \rho \epsilon_{t+1} + \epsilon_{t+2}\\
e_t^F(2) & = Y_{t+2} - Y_t^F(2) = \rho \epsilon_{t+1} + \epsilon_{t+2}
\end{align*}

\end{enumerate}
\end{frame}


%%%%%%%%%%%%%%%%%%%%%%%%%%%%
%%%%%%%%%%%%%%%%%%%%%%%%%%%%
\begin{frame}
\frametitle{Series de tiempo}
\framesubtitle{Pron\'osticos ARIMA: soluci\'on}

\textbf{Ejemplo}: Sea el proceso $Y_{t+1} = c + \rho Y_t + \epsilon_{t+1}$

\begin{enumerate}
\item[5.] Obtenga una expresi\'on general para el pron\'osticos \textbf{h} pasos hacia adelante.

\vspace{-0.8cm}

\begin{align*}
Y_t^F(h) & = c(1+\rho + \rho^2 + \dots + \rho^{h-1}) + \rho^h Y_t
\end{align*}
\item[6.] Calcule el error cuadr\'atico medio de estimaci\'on \textbf{h} pasos hacia adelante.

\vspace{-0.8cm}

\begin{align*}
ECM & = E[e_t^F(h)^2] = \sum_{i=0}^{h-1} E[\rho^{2i}\epsilon_{t+h-i}^2]\\
& = \sum_{i=0}^{h-1} \rho^{2i}\sigma^2 = \frac{\sigma^2[1-\rho^{2h}]}{1-\rho^2}
\end{align*}

\end{enumerate}
\end{frame}


%%%%%%%%%%%%%%%%%%%%%%%%%%%%
%%%%%%%%%%%%%%%%%%%%%%%%%%%%
\begin{frame}
\frametitle{Series de tiempo}
\framesubtitle{Pron\'osticos ARIMA: soluci\'on}

\textbf{Notar} que:
\vspace{0.5cm}
\begin{itemize}
\item 
\vspace{-0.3cm}
\begin{align*}
\lim_{h \longrightarrow \infty} Y_t & = \frac{c}{1-\rho} = E[Y_t]
\end{align*}

\vspace{0.5cm}
\item 
\vspace{-0.5cm}
\begin{align*}
\lim_{h \longrightarrow \infty}  ECM & = \frac{\sigma^2}{1-\rho^2} = V(Y_t)
\end{align*}
\item ECM es creciente en h, pero converge en el largo plazo a la varianza del proceso.
\end{itemize}
\end{frame}




%%%%%%%%%%%%%%%%%%%%%%%%%%%%
%%%%%%%%%%%%%%%%%%%%%%%%%%%%
\begin{frame}
\frametitle{Series de tiempo}
\framesubtitle{Pron\'osticos ARIMA: ejercicios propuestos}

\textbf{Ejercicio 1}: Sea el proceso $Y_{t+1} = \rho Y_t + \epsilon_{t+1} - \theta \epsilon_t$, con $|\rho|<1 $

\vspace{0.5cm}

\begin{enumerate}
\item[1.] Calcule el pron\'ostico un paso hacia adelante del proceso.
\item[2.] Calcule el error de pron\'ostico un paso hacia adelante del proceso.
\item[3.] Obtenga una expresi\'on general para el pron\'ostico h pasos hacia adelante.
\item[4.] Usando la siguiente representaci\'on de $Y_{t+1} = \sum_{i=0}^{\infty} (\rho - \theta) \rho^i \epsilon_{t-i} + \epsilon_{t+1}$, calcule $E[Y_{t+1}]$ y $V[Y_{t+1}]$.
\item[5.] Qu\'e sucede con $Y_t^F(h)$ cuando $h \longrightarrow \infty$?
\end{enumerate}

\end{frame}


%%%%%%%%%%%%%%%%%%%%%%%%%%%%
%%%%%%%%%%%%%%%%%%%%%%%%%%%%
\begin{frame}
\frametitle{Series de tiempo}
\framesubtitle{Pron\'osticos ARIMA: ejercicios propuestos}

\textbf{Ejercicio 2}: Suponga que $X_{t}$ es una caminata aleatoria con deriva e $Y_{t}$ un proceso AR(1) estacionario, ambos independientes entre s\'i. ?`Qu\'e tipo de proceso es $Z_{t}=X_{t}+Y_{t}$?

\vspace{0.5cm}
\textbf{Ejercicio 3}: Demuestre que el siguiente modelo puede formularse como un ARIMA(0,2,2):
    \begin{equation}
        y_t = \mu_t+\epsilon_t
    \end{equation}
    \begin{equation}
        \mu_t = \mu_{t-1}+\beta_{t-1}+\eta_t
    \end{equation}
    \begin{equation}
        \beta_t = \beta_{t-1}+\zeta_t
    \end{equation}
    
    en donde $\epsilon_t$, $\eta_t$ y $\zeta_t$ son ruido blanco, independientes entre s\'i.

\end{frame}

\end{document} 