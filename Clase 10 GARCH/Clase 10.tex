\documentclass{beamer}
\usepackage[document]{ragged2e}
\usepackage[spanish]{babel}
\usepackage[utf8]{inputenc}
\usepackage{lmodern}
\usepackage{amsmath}
\mode<presentation> {
	%\usetheme{default} % liso sin nada bueno simple
	%\usetheme{Pittsburgh} % liso sin nada bueno simple
	%\usetheme{Montpellier} % liso con orden arriba
	%\usetheme{Singapore} % orden arriba liso (mas oscuro)
	%\usetheme{Boadilla} % parecido al liso, con pie de pagina
	
	%\usetheme{Luebeck} % encabezado y pie de pagina (oscuro)
	%\usetheme{Copenhagen} % encabezado y pie de pagina
	\usetheme{Madrid} % encabezado y pie de pagina (azul)
		
	%\usetheme{Goettingen} % con orden a la derecha
	%\usetheme{Marburg} % con orden a la derecha (Oscuro)
	%\usetheme{Hannover} % con orden a la izquierda
	
	%%%%% Colores %%%%%
	%\usecolortheme{albatross} % fondo azul
	%\usecolortheme{beetle} % fondo plomo
		
	%\usecolortheme{dove} % quita todo el color, blanco
	%\usecolortheme{fly} % todo plomo

	%\usecolortheme{crane} % encabezado pie de pagina, naranjo
	%\usecolortheme{seagull} % encabezado pie de pagina, plomo
	\usecolortheme{seahorse} % encabezado pie de pagina azulplomo
	%\usecolortheme{wolverine} % encabezado pie de pagina, amarillo
}
\usepackage{graphicx}
\usepackage{subfigure}
%----------------------------------------------------------------------------------------
%	TITLE PAGE
%----------------------------------------------------------------------------------------

\title[Econometr\'ia Financiera]{Econometr\'ia Financiera}
\subtitle{2022\\
Modelando Volatilidad}
\author[Rodrigo Ortiz] {Rodrigo Ortiz}
\institute[UAI]{
	Universidad Adolfo Ib\'a\~nez \\
}
\date{Santiago, 2022}
%---------------------------------------------------------
% Presentaci�n
%---------------------------------------------------------
\begin{document}
\begin{frame}
	\titlepage
\end{frame}
%---------------------------------------------------------
% Indice
%---------------------------------------------------------
\begin{frame}
	\frametitle{Una excursi\'on al mundo no-lineal}

\begin{itemize}
\item \textbf{Motivaci\'on}: los modelos lineales (y de series de tiempo) estructurales, no pueden explicar una serie de caracter\'isticas importantes comunes a muchos datos financieros
\item Nuestro modelo estructural \textit{tradicional} podr\'ia ser algo como:
\end{itemize}

\begin{align*}
y_t=& \beta_0+\beta_1x_{1t}+\dots+\beta_kx_{kt}+u_t
\end{align*}
Nosotros asumimos que $u_t\sim N(0,\sigma^2)$
\end{frame}

%---------------------------------------------------------
%---------------------------------------------------------
\begin{frame}
	\frametitle{Modelos no lineales: una definici\'on}

\begin{itemize}
\item Campbell, Lo y Mackinlay (1997) definen un proceso de generaci\'on de datos no lineal como uno que se puede escribir como:
\begin{align*}
y_t=& f(u_t,u_{t-1},u_{t-2},\dots)
\end{align*}
donde $u_t$ es un t\'erminos de error aleatorio (iid) y $f$ es una funci\'on no lineal.
\item Tambi\'en dan una definici\'on un poco m\'as espec\'ifica:
\begin{align*}
y_t=& g(u_{t-1},u_{t-2},\dots)+u_t\sigma^2(u_{t-1},u_{t-2},\dots)
\end{align*}
donde $g$ es una funci\'on de t\'erminos de error pasados solamente y $\sigma^2$ es un
t\'ermino de varianza.
\item Modelos con $g()$ no lineales, son \textit{no-lineales en la media}, mientras que los no lineales en $\sigma^2()$ son \textit{no-lineales en varianza}.
\end{itemize}
\end{frame}

%---------------------------------------------------------
%---------------------------------------------------------
\begin{frame}
	\frametitle{Tipos de modelos no lineales}

\begin{itemize}
\item El paradigma lineal es muy \'util. Muchas relaciones aparentemente no lineales se pueden linealizar, a trav\'es de una transformaci\'on adecuada. Por otra parte, es probable que muchas relaciones en finanzas sean intr\'insecamente no lineales.
\item Hay muchos tipos de modelos no lineales, e.g. 
\begin{itemize}
\item ARCH / GARCH
\item switching models
\item bilinear models
\end{itemize}
\end{itemize}
\end{frame}


%---------------------------------------------------------
%---------------------------------------------------------
\begin{frame}
	\frametitle{Pruebas de no linealidad}

\begin{itemize}
\item Las herramientas \textit{tradicionales} de an\'alisis de series temporales (ACF, an\'alisis espectral, etc.) pueden no encontrar evidencia de que podamos utilizar un modelo lineal, pero los datos a\'un pueden no ser independientes.
\item Pruebas de Portmanteau para la dependencia no lineal se han desarrollado. La m\'as simple es el RESET de Ramsey, que tom\'o la forma:
\begin{align*}
\hat{u_t}=& \beta_0+\beta_1\hat{y_t^2}+\beta_2\hat{y_t^3}+\dots+\beta_{p-1}\hat{y_t^p}+\nu_t
\end{align*}
\item Se dispone de muchas otras pruebas de no linealidad, por ejemplo, el \textit{test 
BDS} y la prueba de biespectro.
\item Un modelo no lineal en particular que ha demostrado ser muy \'util en las finanzas es el modelo ARCH, debido a Engle (1982).
\end{itemize}
\end{frame}

%---------------------------------------------------------
%---------------------------------------------------------
\begin{frame}
	\frametitle{Heteroscedasticidad Revisada}

\begin{itemize}
\item Un ejemplo de un modelo estructural es
\begin{align*}
y_t=& \beta_1+\beta_2x_{2t}+\beta_3x_{3t}+\beta_4x_{4t}+u_t
\end{align*}

con $u_t\sim N(0,\sigma_u^2)$
\item La suposici\'on de que la varianza de los errores es constante se conoce como homocedasticidad, i.e. $Var(u_t)=\sigma_u^2$
\item ?`Qu\'e pasa si la varianza de los errores no es constante?
\begin{itemize}
\item heterocedasticidad
\item Lo que implican que las estimaciones de error est\'andar podr\'ia estar equivocadas.
\end{itemize}
\item ?`En la pr\'actica es la varianza de los errores constante en el tiempo? \textbf{No} para los datos financieros.
\end{itemize}
\end{frame}


%---------------------------------------------------------
%---------------------------------------------------------
\begin{frame}
	\frametitle{Autoregressive Conditionally Heteroscedastic (ARCH) Models}

\begin{itemize}
\item Utilicemos un modelo que no asume que la varianza es constante.
\item Recuerde la definici\'on de la varianza de $u_t$:
\begin{align*}
\sigma_t^2&=Var(u_t/u_{t-1},u_{t-2},\dots)=E[(u_t-E(u_t))^2/u_{t-1},u_{t-2},\dots]
\end{align*}
Normalmente suponemos que $E(u_t)=0$, de aqu\'i:
\begin{align*}
\sigma_t^2&=Var(u_t/u_{t-1},u_{t-2},\dots)=E[u_t^2/u_{t-1},u_{t-2},\dots]
\end{align*}
\item ?`De qu\'e depender\'a el valor actual de la varianza de los errores?
\begin{itemize}
\item Cuadrado t\'erminos de error anteriores.
\item Esto lleva al modelo the autoregressive conditionally heteroscedastic model \textit{condicionalmente heteroced\'astico autorregresivo} para la varianza de los errores
\begin{align*}
\sigma_t^2&=\alpha_0+\alpha_1u_{t-1}^2
\end{align*}
\item Este se conoce como: ARCH(1) model.
\end{itemize}
\end{itemize}
\end{frame}

%---------------------------------------------------------
%---------------------------------------------------------
\begin{frame}
	\frametitle{Autoregressive Conditionally Heteroscedastic (ARCH) Models (cont.)}

\begin{itemize}
\item El modelo completo ser\'ia:

\begin{align*}
y_t=& \beta_1+\beta_2x_{2t}+\beta_3x_{3t}+\beta_4x_{4t}+u_t
\end{align*}
con $u_t \sim N(0,\sigma_t^2)$, donde $\sigma_t^2=\alpha_0+\alpha_1u_{t-1}^2$
\item Podemos extender f\'acilmente esto para el caso general en que la varianza del error depende de $q$ rezagos de errores al cuadrado:
\begin{align*}
\sigma_t^2 & =\alpha_0+\alpha_1 u_{t-1}^2+\alpha_2 u_{t-2}^2+\dots+\alpha_q u_{t-q}^2
\end{align*}
\item Es es un ARCH(q) model.
\item En lugar de llamar a la varianza, $\sigma_t^2$ en la literatura se le suele llamar
$h_t$, por lo que el modelo es en definitiva:
\begin{align*}
y_t=& \beta_1+\beta_2x_{2t}+\beta_3x_{3t}+\beta_4x_{4t}+u_t
\end{align*}
con $u_t \sim N(0,h_t)$, donde $h_t=\alpha_0+\alpha_1 u_{t-1}^2+\alpha_2 u_{t-2}^2+\dots+\alpha_q u_{t-q}^2$
\end{itemize}
\end{frame}


%---------------------------------------------------------
%---------------------------------------------------------
\begin{frame}
	\frametitle{Otra forma de representar a los ARCH Models}

\begin{itemize}
\item Por ejemplo, considere un ARCH (1). En lugar de la representaci\'on anterior, podemos escribir
\begin{align*}
y_t=& \beta_1+\beta_2x_{2t}+\beta_3x_{3t}+\beta_4x_{4t}+u_t
\end{align*}
con $u_t=\nu_t\sigma_t$
\begin{align*}
\sigma_t&=\sqrt{\alpha_0+\alpha_1u_{t-1}^2}
\end{align*}
con $\nu_t \sim N(0,1)$
\item Las dos formas representas diferentes maneras de expresar exactamente el mismo modelo. La primera forma es m\'as f\'acil de entender, mientras que la segunda representa mejor la simulaci\'on de un modelo ARCH.
\end{itemize}
\end{frame}



%---------------------------------------------------------
%---------------------------------------------------------
\begin{frame}
	\frametitle{Las pruebas de \textit{efectos ARCH}}

\begin{enumerate}
\item [1] En primer lugar, ejecutar una regresi\'on lineal, por ejemplo,
\begin{align*}
y_t=& \beta_1+\beta_2x_{2t}+\dots+\beta_kx_{kt}+u_t
\end{align*}
guardando los residuos, $\hat{u_t}$
\item [2] A continuaci\'on, elevar los residuos al cuadrado, y correr una regresi\'on sobre los $q$ rezagos propios para la prueba de ARCH de orden $q$, es decir, ejecutar la regresi\'on
\begin{align*}
\hat{u_t}^2=& \gamma_0+\gamma_1\hat{u}^2_{t-1}+\gamma_2\hat{u}^2_{t-2}+\dots+\gamma_{q}\hat{u}^2_{t-q}+\nu_t
\end{align*}
donde $\nu_t$ es iid.
Obtener $R^2$ de esta regresi\'on
\item [3] El test estad\'istico se define como $TR^2$ (el n\'umero de observaciones multiplicado por el coeficiente de correlaci\'on m\'ultiple) a partir de la \'ultima regresi\'on, y se distribuye como una $\chi^2(q)$
\end{enumerate}
\end{frame}


%---------------------------------------------------------
%---------------------------------------------------------
\begin{frame}
	\frametitle{Las pruebas de \textit{efectos ARCH}}

\begin{enumerate}
\item [4] Las hip\'otesis nula y alternativa son
\end{enumerate}

\begin{align*}
H_0:& \gamma_1=0 & \textup{and} \hspace*{0.6cm} & \gamma_2=0 & \textup{and} \hspace*{0.6cm} & \dots & \textup{and} \hspace*{0.6cm} \gamma_q=0\\
H_1:& \gamma_1 \neq 0 & \textup{or} \hspace*{0.6cm} & \gamma_2 \neq 0 & \textup{or} \hspace*{0.6cm} & \dots &  \textup{or} \hspace*{0.6cm} \gamma_q \neq 0 
\end{align*}

Si el valor de la prueba estad\'istica es mayor que el valor cr\'itico de la distribuci\'on $\chi^2(q)$, se rechaza la hip\'otesis nula.
\vspace{0.5cm}

Tenga en cuenta que la prueba ARCH tambi\'en a veces se aplica directamente a la rentabilidad, en lugar de los residuos de la etapa 1 anterior.
\end{frame}

%---------------------------------------------------------
%---------------------------------------------------------
\begin{frame}
	\frametitle{Problemas con ARCH(q) Models}

\begin{itemize}
\item ?`C\'omo decidimos el mejor q?
\vspace{0.5cm}
\item El valor requerido de q podr\'ia ser muy grande.
\vspace{0.5cm}
\item Las restricciones de no negatividad pueden ser violadas.
\vspace{0.5cm}
\item Cuando se estima un modelo ARCH, requerimos $\alpha_i > 0$  $\forall i=1,2,...,q$ (ya que la varianza no puede ser negativa)
\vspace{0.5cm}
\item Una extensi\'on natural de un modelo ARCH(q), que evita algunos de estos problemas es el modelo GARCH.
\end{itemize}
\end{frame}

%---------------------------------------------------------
%---------------------------------------------------------
\begin{frame}
	\frametitle{Generalised ARCH (GARCH) Models}

\begin{itemize}
\item Debido a Bollerslev (1986). Deja que la varianza condicional sea dependiente de sus propios rezagos anteriores
\item La ecuaci\'on de la varianza es ahora
\begin{align*}
\sigma_t^2=&\alpha_0+\alpha_1u^2_{t-1}+\beta\sigma^2_{t-1}
\end{align*}
\item Se trata de un GARCH (1,1), que es como un ARMA (1,1) de la ecuaci\'on
de la varianza.
\item Tambi\'en podr\'iamos escribir
\begin{align*}
\sigma^2_{t-1}=&\alpha_0+\alpha_1u^2_{t-2}+\beta\sigma^2_{t-2} \\
\sigma^2_{t-2}=&\alpha_0+\alpha_1u^2_{t-3}+\beta\sigma^2_{t-3} \\
\end{align*}
\item Sustituyendo en primera ecuaci\'on por $\sigma^2_{t-1}$
\begin{align*}
\sigma_t^2=&\alpha_0+\alpha_1u^2_{t-1}+\beta(\alpha_0+\alpha_1u^2_{t-2}+\beta\sigma^2_{t-2})\\
\sigma_t^2=&\alpha_0+\alpha_1u^2_{t-1}+\beta\alpha_0+\beta\alpha_1u^2_{t-2}+\beta^2\sigma^2_{t-2}
\end{align*}
\end{itemize}
\end{frame}


%---------------------------------------------------------
%---------------------------------------------------------
\begin{frame}
	\frametitle{Generalised ARCH (GARCH) Models}

\begin{itemize}
\item Sustituyendo por $\sigma^2_{t-2}$
\begin{align*}
\sigma_t^2=&\alpha_0+\alpha_1u^2_{t-1}+\beta\alpha_0+\beta\alpha_1u^2_{t-2}+\beta^2\sigma^2_{t-2}\\
\sigma_t^2=&\alpha_0+\alpha_1u^2_{t-1}+\beta\alpha_0+\beta\alpha_1u^2_{t-2}+\beta^2(\alpha_0+\alpha_1u^2_{t-3}+\beta\sigma^2_{t-3})\\
\sigma_t^2=&\alpha_0+\alpha_1u^2_{t-1}+\beta\alpha_0+\beta\alpha_1u^2_{t-2}+\beta^2\alpha_0+\beta^2\alpha_1u^2_{t-3}+\beta^3\sigma^2_{t-3}
\end{align*}
\item Un n\'umero infinito de sustituciones sucesivas producir\'ia
\begin{align*}
\sigma_t^2=&\alpha_0(1+\beta+\beta^2+\dots)+\alpha_1 u^2_{t-1}(1+\beta L+\beta^2 L^2+\dots)+\beta^{\infty}\sigma^2_0
\end{align*}
\item As\'i que el GARCH(1,1) puede escribirse como un modelo de ARCH orden infinito.
\item Nuevamente, podemos extender el GARCH(1,1) a un GARCH(p,q):
\begin{align*}
\sigma_t^2=&\alpha_0+\alpha_1u^2_{t-1}+\dots+\alpha_qu^2_{t-q}+\beta_1\sigma^2_{t-1}+\dots+\beta_p\sigma^2_{t-p}\\
\sigma_t^2=&\alpha_0+\sum_{i=1}^{q}\alpha_iu^2_{t-i}+\sum_{j=1}^{p}\beta_j\sigma^2_{t-j}
\end{align*}
\end{itemize}
\end{frame}



%---------------------------------------------------------
%---------------------------------------------------------
\begin{frame}
	\frametitle{Generalised ARCH(GARCH) Models (cont)}

\begin{itemize}
\item  Pero, en general, un modelo GARCH(1,1) ser\'a suficiente para capturar la volatilidad clusterizada de los datos.
\item ?`Por qu\'e es mejor GARCH que ARCH?
\begin{itemize}
\item M\'as parsimonioso - evita el sobreajuste
\item Menos probabilidades de violar restricciones de no negatividad
\end{itemize}
\end{itemize}
\end{frame}

%---------------------------------------------------------
%---------------------------------------------------------
\begin{frame}
	\frametitle{La varianza incondicional bajo la especificaci\'on GARCH}

\begin{itemize}
\item  La varianza incondicional de $u_t$ esta dada por:
\begin{align*}
Var(u_t)=&\frac{\alpha_0}{1-(\alpha_1+\beta)}
\end{align*}
cuando $\alpha_1+\beta<1$
\item $\alpha_1+\beta\geq 1$ que se denomina \textit{no estacionariedad} en la varianza
\item $\alpha_1+\beta=1$ que se denomina intergrated GARCH
\item Para varianza no estacionaria, los pron\'osticos de la varianza condicional no converger\'an a su valor incondicional en la medida que el tiempo aumenta.
\end{itemize}
\end{frame}


%---------------------------------------------------------
%---------------------------------------------------------
\begin{frame}
	\frametitle{Estimaci\'on de modelos ARCH/GARCH}

\begin{itemize}
\item  Dado que el modelo ya no es de la forma lineal que acostumbramos, no podemos usar MCO.
\vspace{0.3cm}
\item Utilizamos otra t\'ecnica conocida como de m\'axima verosimilitud.
\vspace{0.3cm}
\item El m\'etodo funciona mediante la b\'usqueda de los valores m\'as probables de los par\'ametros, dados los datos reales.
\vspace{0.3cm}
\item M\'as espec\'ificamente, formamos una funci\'on de verosimilitud y la maximizamos.
\end{itemize}
\end{frame}


%---------------------------------------------------------
%---------------------------------------------------------
\begin{frame}
	\frametitle{Estimaci\'on de modelos ARCH/GARCH}

\begin{itemize}
\item Los pasos a seguir en la estimaci\'on de un modelo ARCH o GARCH son los siguientes:
\begin{enumerate}
\item Especificar las ecuaciones apropiadas para la media y la varianza - por ejemplo, un AR(1)-GARCH(1,1):

\begin{align*}
y_t=& \mu +\phi y_{t-1}+ u_t \\
\sigma_t^2=&\alpha_0+\alpha_1u^2_{t-1}+\beta_1\sigma^2_{t-1}
\end{align*}
donde $u_t \sim N(0,\sigma^2)$

\item Especifique la funci\'on de verosimilitud para maximizar: 

\begin{align*}
L=& -\frac{T}{2} log(2\pi)-\frac{1}{2}\sum_{i=1}^{T}log (\sigma^2_{t})-\frac{1}{2}\sum_{j=1}^{T}\frac{(y_i - \mu -\phi y_{t-1})^2}{\sigma^2_{t}}
\end{align*}

\item El computador maximizar la funci\'on, y calcula los par\'ametros y sus errores est\'andar \dots
\end{enumerate}
\end{itemize}
\end{frame}

%---------------------------------------------------------
%---------------------------------------------------------
\begin{frame}
	\frametitle{Extensiones al modelo GARCH B\'asico}

\begin{itemize}
\item Desde que se desarroll\'o el modelo GARCH, se han propuesto un gran n\'umero de extensiones y variantes. Tres de los ejemplos m\'as importantes son los modelos GARCH-M, EGARCH, y GJR.
\vspace{0.4cm}
\item Problemas con los modelos GARCH(p,q):
\begin{itemize}
\item Restricciones de no negatividad pueden ser violadas
\item Los Modelos GARCH no pueden dar cuenta de los efectos de apalancamiento
\end{itemize}
\vspace{0.4cm}
\item Posibles soluciones: el modelo GARCH exponencial (EGARCH) o el modelo GJR, que plantean modelos GARCH asim\'etricos.
\end{itemize}
\end{frame}

%---------------------------------------------------------
%---------------------------------------------------------
\begin{frame}
	\frametitle{El Modelo EGARCH}

\begin{itemize}
\item Sugerido por Nelson (1991). La ecuaci\'on de varianza est\'a dada por:
\begin{align*}
log (\sigma^2_{t})=& \omega+ \beta log (\sigma^2_{t-1})+\gamma\frac{u_{t-1}}{\sqrt{\sigma^2_{t-1}}}+\alpha  \left[ \frac{|u_{t-1}|}{\sqrt{\sigma^2_{t-1}}} - \sqrt{\frac{2}{\pi}}\right]
\end{align*}
\item Ventajas del modelo
\begin{itemize}
\item Dado que modelamos $log(\sigma^2_{t})$, incluso si los par\'ametros son negativos, $\sigma^2_{t}$ ser\'a positivo.
\item Podemos tomar cuenta el efecto de apalancamiento: si la relaci\'on entre volatilidad y rentabilidad es negativa, $\gamma$, ser\'a negativo.
\end{itemize}
\end{itemize}
\end{frame}


%---------------------------------------------------------
%---------------------------------------------------------
\begin{frame}
	\frametitle{El Modelo GJR}

\begin{itemize}
\item Debido a Glosten, Jaganathan and Runkle (1993):
\begin{align*}
\sigma^2_{t}=&\alpha_0+\alpha_1 u^2_{t-1}+\beta \sigma^2_{t-1}+\gamma u^2_{t-1} I_{t-1}
\end{align*}
donde $I_{t-1}=1$ si $u_{t-1}<0$ y  $I_{t-1}=0$ si $u_{t-1}>0$
\vspace{0.4cm}
\item Para un efecto de apalancamiento ver\'iamos $\gamma >0$
\vspace{0.4cm}
\item Requerimos $\alpha_1 + \gamma \geq 0$ y $\alpha_1 \geq 0$ para no-negatividad.
\end{itemize}
\end{frame}

%---------------------------------------------------------
%---------------------------------------------------------
\begin{frame}
	\frametitle{GARCH - in Mean}

\begin{itemize}
\item Si esperamos que un riesgo se compense con una mayor rentabilidad, ?`por qu\'e no dejar que el retorno de un valor determinado sea parcialmente determinado por su riesgo?
\item Engle, Lilien y Robins (1987) sugirieron la especificaci\'on ARCH-M. Un modelo GARCH-M ser\'ia
\begin{align*}
y_t=&\mu+\delta \sigma_{t-1}+ u_{t} \\
\sigma^2_{t}=& \alpha_0+\alpha_1 u^2_{t-1}+\beta \sigma^2_{t-1}
\end{align*}
donde $u_t \sim N(0,\sigma^2_{t})$
\vspace{0.4cm}
\item $\delta$ puede interpretarse como una especie de prima por riesgo.
\vspace{0.4cm}
\item Es posible combinar todos o algunos de estos modelos en conjunto para obtener modelos m\'as complejos, \textit{h\'ibridos} - por ejemplo, un modelo ARMA-EGARCH(1,1)-M.
\end{itemize}
\end{frame}

%---------------------------------------------------------
%---------------------------------------------------------
\begin{frame}
	\frametitle{?`Por qu\'e usar modelos GARCH?}

\begin{itemize}
\item GARCH puede modelar el efecto del agrupamiento de volatilidad, ya que la varianza condicional es autorregresiva. Tales modelos pueden ser utilizados para pronosticar la volatilidad.
\item Vimos que
\begin{align*}
Var(y_t \ y_{t-1},y_{t-2},\dots)=& Var(u_t \ u_{t-1},u_{t-2},\dots)
\end{align*}
\item Modelado $\sigma^2_t$ nos permitir\'a modelar y pronosticar $y_t$ tambi\'en.
\item Los pron\'osticos de varianza son aditivos en el tiempo.
\end{itemize}
\end{frame}


%---------------------------------------------------------
%---------------------------------------------------------
\begin{frame}
	\frametitle{Modelos GARCH Multivariados}

\begin{itemize}
\item Los modelos GARCH multivariantes (MGARCH) generalizan los modelos GARCH univariantes y permiten incorporar relaciones entre los procesos de volatilidad de varias series. Queremos saber c\'omo los cambios en la volatilidad de una acci\'on afectan a la volatilidad de otra acci\'on. Estas relaciones se pueden parametrizar de varias maneras. El comando de Stata \textbf{mgarch} implementa cuatro parametrizaciones com\'unmente utilizadas:
\begin{enumerate}
\item el modelo vech diagonal (mgarch dvech),
\item el modelo de correlaci\'on condicional constante (mgarch ccc),
\item el modelo de correlaci\'on condicional din\'amica (mgarch dcc), y
\item el modelo de correlaci\'on condicional variable (mgarch vcc).
\end{enumerate}
\end{itemize}
\end{frame}

%---------------------------------------------------------
%---------------------------------------------------------
\begin{frame}
	\frametitle{Referencias}

\begin{itemize}
\item Campbell, J. Y., Lo, A. W. and MacKinlay, A. C. (1997) The Econometrics of Financial Markets, Princeton University Press, Princeton, NJ
\item Engle, R. F. (1982) Autoregressive Conditional Heteroskedasticity with Estimates of the Variance of United Kingdom Inflation, Econometrica 50(4), 9871007
\item Bollerslev, T. (1986) Generalised Autoregressive Conditional Heteroskedasticity, Journal of Econometrics 31, 307?27
\item Nelson, D. B. (1991) Conditional Heteroskedasticity in Asset Returns: a New Approach,
Econometrica 59(2), 347?70
\item Glosten, L. R., Jagannathan, R. and Runkle, D. E. (1993) On the Relation Between the
Expected Value and the Volatility of the Nominal Excess Return on Stocks, The Journal
of Finance 48(5), 1779?801
\item Engle, R. F., Lilien, D. M. and Robins, R. P. (1987) Estimating Time Varying Risk Premia in the Term Structure: the ARCH-M Model, Econometrica 55(2), 391?407
\end{itemize}
\end{frame}


\end{document} 