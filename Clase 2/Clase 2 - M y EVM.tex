\documentclass{beamer}
\usepackage[document]{ragged2e}
\usepackage[spanish]{babel}
\usepackage[utf8]{inputenc}
\usepackage{lmodern}
\usepackage{amsmath}
\mode<presentation> {
	%\usetheme{default} % liso sin nada bueno simple
	%\usetheme{Pittsburgh} % liso sin nada bueno simple
	%\usetheme{Montpellier} % liso con orden arriba
	%\usetheme{Singapore} % orden arriba liso (mas oscuro)
	%\usetheme{Boadilla} % parecido al liso, con pie de pagina
	
	%\usetheme{Luebeck} % encabezado y pie de pagina (oscuro)
	%\usetheme{Copenhagen} % encabezado y pie de pagina
	\usetheme{Madrid} % encabezado y pie de pagina (azul)
		
	%\usetheme{Goettingen} % con orden a la derecha
	%\usetheme{Marburg} % con orden a la derecha (Oscuro)
	%\usetheme{Hannover} % con orden a la izquierda
	
	%%%%% Colores %%%%%
	%\usecolortheme{albatross} % fondo azul
	%\usecolortheme{beetle} % fondo plomo
		
	%\usecolortheme{dove} % quita todo el color, blanco
	%\usecolortheme{fly} % todo plomo

	%\usecolortheme{crane} % encabezado pie de pagina, naranjo
	%\usecolortheme{seagull} % encabezado pie de pagina, plomo
	\usecolortheme{seahorse} % encabezado pie de pagina azulplomo
	%\usecolortheme{wolverine} % encabezado pie de pagina, amarillo
}
\usepackage{graphicx}
\usepackage{subfigure}
%----------------------------------------------------------------------------------------
%	TITLE PAGE
%----------------------------------------------------------------------------------------

\title[Econometr\'ia Financiera]{Econometr\'ia Financiera}
\subtitle{2023\\
Tema 2: Estimaci\'on por momentos y m\'axima verosimilitud}
\author[Rodrigo Ortiz] {Rodrigo Ortiz}
\institute[UAH]{
	Universidad Adolfo Ib\'a\~nez \\
}
\date{Chile, 2023}
%---------------------------------------------------------
% Presentaci�n
%---------------------------------------------------------
\begin{document}
\begin{frame}
	\titlepage
\end{frame}
%---------------------------------------------------------
% Indice
%---------------------------------------------------------
\begin{frame}
	\frametitle{Planificaci\'on}
	\framesubtitle{Agenda}
	\tableofcontents
\end{frame}

\section{Definiciones importantes}
%%%%%%%%%%%%%%%%%%%%%%%%%%%%
%%%%%%%%%%%%%%%%%%%%%%%%%%%%
\begin{frame}
\frametitle{Definiciones importantes}

\textbf{Definici\'on}: Sea $X_1,\dots,X_n$, n variables aleatorias con funci\'on de probabilidad conjunta $f(x_1,\dots,x_n$ y funciones marginales $f_1(x_1),\dots, f_n(x_n)$

\begin{itemize}
  \item $f(x_1,\dots,x_n)=f_1(x_1)f_2(x_2),\dots,f_n(x_n)$
  \item $E[x_1x_2\dots x_n]=E[x_1]E[x_2]\dots E[x_n]$
\end{itemize}

\vspace{0.5cm}

\textbf{Definici\'on}: Sea dice que las n variables aleatorias son id\'enticamente distrubuidas
si y s\'olo si:

\begin{itemize}
  \item $f_1(x_1)=f_2(x_2)=\dots=f_n(x_n)$
\end{itemize}
\end{frame}

%%%%%%%%%%%%%%%%%%%%%%%%%%%%
%%%%%%%%%%%%%%%%%%%%%%%%%%%%
\begin{frame}
\frametitle{Definiciones importantes}

\textbf{Definici\'on}: Sea $X_1,\dots,X_n$, n variables aleatorias conjunto de n variables aleatorias
$(x_1,\dots,x_n)$ constituye una muestra aleatoria si y s\'olo si:

\begin{itemize}
  \item $X_1,\dots,X_n$ son variables aleatorias estoc\'asticamentte  independientes
  \item $X_1,\dots,X_n$ sin variables aleatorias id\'enticamente distribuidas
\end{itemize}

\vspace{0.5cm}

\textbf{Definici\'on}: Sea $X$ variable aleatoria, se define el k-\'esimo momento
problacional del origen $\mu_k$

\begin{itemize}
  \item [] $\mu_k=E[x^k]$
\end{itemize}

\vspace{0.5cm}

\textbf{Definici\'on}: Sea $X$ variable aleatoria, se define el k-\'esimo momento
problacional alrededor de $\mu$, $\mu'_k$

\begin{itemize}
  \item [] $\mu'_k=E[(x-\mu)^k]$
\end{itemize}
\end{frame}

%%%%%%%%%%%%%%%%%%%%%%%%%%%%
%%%%%%%%%%%%%%%%%%%%%%%%%%%%
\begin{frame}
\frametitle{Definiciones importantes}

\textbf{Definici\'on}: Se define el k-\'esimo momento muestral alrededor del origen, $M_k$,
mediante $M_k=\frac{1}{n}\sum_{1=1}^{n}x_i^k$, donde $x_1,\dots,x_n$ constituye una muetsra aleatoria
de esta poblaci\'on

\vspace{0.5cm}

\textbf{Definici\'on}: Sea $x_1,\dots,x_n$ una muestra aleatoria de una poblaci\'on $X$ se define el
k-\'esimo momento muestral alrededor de $\overline{x}$, $M_k'$, mediante

\begin{itemize}
  \item [] $M_k'=E[(x_i-\overline{x})^k]$
\end{itemize}
\end{frame}


\begin{frame}
\frametitle{Definiciones importantes: Ejercicios}

\textbf{Ejercicio 1}: Muestre que $E[M_k]=\mu_k$ y $V(M_K)=E\left[\left( \frac{1}{n} \sum_{i=1}^{n}x_i^k\right)^2\right]-\mu_k^2$

\vspace{0.5cm}

\textbf{Ejercicio 2}:Sea $x_1,\dots,x_n$ una muestra aleatoria de una poblaci\'on X. Determine el $E[M_2']$
\end{frame}

\section{M\'etodos de estimaci\'on: M\'etodo de los Momentos}

\begin{frame}
\frametitle{M\'etodos de estimaci\'on: M\'etodo de los Momentos}

Sea $X\thicksim f(x,\theta_1,\dots,\theta_k)$ donde $\theta_1,\dots,\theta_k$ son $k$ par\'ametros desconocidos y sea $(x_1,\dots,x_n)$
una muestra aleatoria de esta poblaci\'on.

\vspace{0.3cm}

Recordemos que el r-\'esimo momento poblacional alrededor del origen es $\mu_1=E[x^n]$ y el r-\'esimo momento muestral alrededor
del origen $M_r=\frac{1}{n}\sum_{i=1}^{n}x_i^r$
\vspace{0.3cm}

Para encontrar los estimadores por mommentos de $\theta_1,\dots,\theta_k$ se procede a formar un sistema de k-ecuaciones
\vspace{0.3cm}

\end{frame}


\begin{frame}
\frametitle{M\'etodos de estimaci\'on: M\'etodo de los Momentos}

\begin{align*}
\mu_1&=M_1\\
\mu_2&=M_2\\
\vdots & \\
\mu_k & = M_k
\end{align*}

Al resolver para $\theta_1,\dots,\theta_k$ encontramos los estimadores por momentos
$\hat{\theta_1},\dots,\hat{\theta_k}$
\end{frame}

\begin{frame}
\frametitle{M\'etodos de estimaci\'on: M\'etodo de los Momentos}
\textbf{Ejercicio 3}: El tiempo en X segundos que un computador tarda en ejecutar una tarea sigue una variable aletaroria
con funci\'on de densidad

\begin{align*}
f(X;\theta)&=\frac{\theta2^\theta}{X^{\theta+1}}, x \geqslant 2, \theta>0
\end{align*}

encuentre el estimador por momentos para $\theta$.

\end{frame}

\section{M\'etodos de estimaci\'on: Estimaci\'on por m\'axima verosimilitud}

\begin{frame}
\frametitle{M\'etodos de estimaci\'on: Estimaci\'on por m\'axima verosimilitud(EMV, MLE ingl\'es)}

Sea $X\thicksim f(x,\theta_1,\dots,\theta_k)$ donde $\theta_1,\dots,\theta_k$ son $k$ par\'ametros desconocidos y sea $(x_1,\dots,x_n)$
una muestra aleatoria de esta poblaci\'on.

\vspace{0.3cm}

Se define la funci\'on de verosimilitud de la muestra aleatoria de los par\'ametros


\begin{align*}
L(X_1,\dots,X_n;\theta_1,\dots,\theta_k)&=f(X_1;\theta_1,\dots,\theta_k)\dots f(X_n;\theta_1,\dots,\theta_k)\\
& = \prod_{i=1}^{n} f(X_i;\theta_1,\dots,\theta_k)
\end{align*}

\end{frame}


\begin{frame}
\frametitle{M\'etodos de estimaci\'on: Estimaci\'on por m\'axima verosimilitud (EMV, MLE ingl\'es)}

Del c\'alculo sabemos que para maximizar una funci\'on obtebemos las derivadas parciales respecto de
$\theta_1,\dots,\theta_k$ e igualamos a cero y resolvemos el sistema

\begin{align*}
\frac{\partial L(X_1,\dots,X_n;\theta_1,\dots,\theta_k)}{\partial \theta_1} & =0 \\
\vdots & \\
\frac{\partial L(X_1,\dots,X_n;\theta_1,\dots,\theta_k)}{\partial \theta_k} & =0 \\
\end{align*}
\end{frame}



\begin{frame}
\frametitle{M\'etodos de estimaci\'on: Estimaci\'on por m\'axima verosimilitud (EMV, MLE ingl\'es)}

$L(X_1,\dots,X_n;\theta_1,\dots,\theta_k)$ en una funci\'on positiva, por lo que el m\'aximo se logra en el mismo punto que
el de la funci\'on $ln L(X_1,\dots,X_n;\theta_1,\dots,\theta_k)$

\vspace{0.3cm}

Resolviendo el sistema:

\begin{align*}
\frac{\partial ln L(X_1,\dots,X_n;\theta_1,\dots,\theta_k)}{\partial \theta_1} & =0 \\
\vdots & \\
\frac{\partial ln L(X_1,\dots,X_n;\theta_1,\dots,\theta_k)}{\partial \theta_k} & =0 \\
\end{align*}


Del \textbf{Ejercicio 3} encontrar el EMV de $\theta$.
\end{frame}
\end{document} 