\documentclass{beamer}
\usepackage[document]{ragged2e}
\usepackage[spanish]{babel}
\usepackage[utf8]{inputenc}
\usepackage{lmodern}
\usepackage{amsmath}
\mode<presentation> {
	%\usetheme{default} % liso sin nada bueno simple
	%\usetheme{Pittsburgh} % liso sin nada bueno simple
	%\usetheme{Montpellier} % liso con orden arriba
	%\usetheme{Singapore} % orden arriba liso (mas oscuro)
	%\usetheme{Boadilla} % parecido al liso, con pie de pagina
	
	%\usetheme{Luebeck} % encabezado y pie de pagina (oscuro)
	%\usetheme{Copenhagen} % encabezado y pie de pagina
	\usetheme{Madrid} % encabezado y pie de pagina (azul)
		
	%\usetheme{Goettingen} % con orden a la derecha
	%\usetheme{Marburg} % con orden a la derecha (Oscuro)
	%\usetheme{Hannover} % con orden a la izquierda
	
	%%%%% Colores %%%%%
	%\usecolortheme{albatross} % fondo azul
	%\usecolortheme{beetle} % fondo plomo
		
	%\usecolortheme{dove} % quita todo el color, blanco
	%\usecolortheme{fly} % todo plomo

	%\usecolortheme{crane} % encabezado pie de pagina, naranjo
	%\usecolortheme{seagull} % encabezado pie de pagina, plomo
	\usecolortheme{seahorse} % encabezado pie de pagina azulplomo
	%\usecolortheme{wolverine} % encabezado pie de pagina, amarillo
}
\usepackage{graphicx}
\usepackage{subfigure}
%----------------------------------------------------------------------------------------
%	TITLE PAGE
%----------------------------------------------------------------------------------------

\title[Econometr\'ia Financiera]{Econometr\'ia Financiera}
\subtitle{2023\\
Tema 2: Esperanza matem\'atica}
\author[Rodrigo Ortiz] {Rodrigo Ortiz}
\institute[UAH]{
	Universidad Adolfo Ib\'a\~nez \\
}
\date{Chile, 2023}
%---------------------------------------------------------
% Presentaci�n
%---------------------------------------------------------
\begin{document}
\begin{frame}
	\titlepage
\end{frame}
%---------------------------------------------------------
% Indice
%---------------------------------------------------------
\begin{frame}
	\frametitle{Planificaci\'on}
	\framesubtitle{Agenda}
	\tableofcontents
\end{frame}

\section{Tema 2: Esperanza matem\'atica}
\subsection{Esperanza, varianza y covarianzas de variables aleatorias}
\subsubsection{Variables discretas}
%%%%%%%%%%%%%%%%%%%%%%%%%%%%
%%%%%%%%%%%%%%%%%%%%%%%%%%%%
\begin{frame}
\frametitle{Esperanza, varianza y covarianzas de variables aleatorias}
\framesubtitle{Variables discretas: Funci\'on de distribuci\'on de probabilidad}

La \textbf{funci\'on de distribuci\'on de probabilidad}, $P(x )$, de una variable aleatoria discreta X expresa
la probabilidad de que X tome el valor x, como una funci\'on de x. Es decir,
\begin{equation*}
  P(x)=P(X=x)
\end{equation*}
para todos los valores de x.
\end{frame}

%%%%%%%%%%%%%%%%%%%%%%%%%%%%
%%%%%%%%%%%%%%%%%%%%%%%%%%%%
\begin{frame}
\frametitle{Esperanza, varianza y covarianzas de variables aleatorias}
\framesubtitle{Variables discretas: Propiedades que deben satisfacer las funciones de probabilidad
de variables aleatorias discretas}

Sea X una variable aleatoria discreta que tiene una funci\'on de probabilidad $P(x)$. En ese caso,
\begin{itemize}
  \item [1.] $0\leq P(x)\leq 1$ para cualquier valor x y
  \item [2.] Las probabilidades individuales suman 1, es decir,
\end{itemize}

\begin{equation*}
  \sum_{x}P(x)=1
\end{equation*}

donde la notaci\'on indica que el sumatorio abarca todos los valores posibles de x.
\end{frame}

%%%%%%%%%%%%%%%%%%%%%%%%%%%%
%%%%%%%%%%%%%%%%%%%%%%%%%%%%
\begin{frame}
\frametitle{Esperanza, varianza y covarianzas de variables aleatorias}
\framesubtitle{Variables discretas: Funci\'on de probabilidad acumulada}

La \textbf{funci\'on de probabilidad acumulada}, $F(x_0)$, de una variable aleatoria X, expresa la probabilidad
de que X no tenga un valor superior a $x_0$, como una funci\'on de $x_0$. Es decir,

\begin{equation*}
  F(x_0)=P(X\leq x_0)
\end{equation*}

donde la funci\'on se eval\'ua en todos los valores de $x_0$.
\end{frame}

%%%%%%%%%%%%%%%%%%%%%%%%%%%%
%%%%%%%%%%%%%%%%%%%%%%%%%%%%
\begin{frame}
\frametitle{Esperanza, varianza y covarianzas de variables aleatorias}
\framesubtitle{Variables discretas: Relaci\'on entre la funci\'on de probabilidad y la funci\'on
de probabilidad acumulada}

Sea X una variable aleatoria que tiene la funci\'on de probabilidad $P(x)$ y la funci\'on de probabilidad
acumulada $F(x_0)$. Podemos demostrar que

\begin{equation*}
  F(x_0)=\sum_{x\leq x_0}P(x)
\end{equation*}

donde la notaci\'on implica que el sumatorio abarca todos los valores posibles de x que son menores
o iguales que $x_0$.
\end{frame}


%%%%%%%%%%%%%%%%%%%%%%%%%%%%
%%%%%%%%%%%%%%%%%%%%%%%%%%%%
\begin{frame}
\frametitle{Esperanza, varianza y covarianzas de variables aleatorias}
\framesubtitle{Variables discretas: Valor esperado}

El \textbf{valor esperado}, $E(X)$, de una variable aleatoria discreta X se define de la forma siguiente:

\begin{equation*}
  E(X)=\mu=\sum_{x}xP(x)
\end{equation*}

donde la notaci\'on indica que el sumatorio abarca todos los valores posibles de x.
\vspace{0.3cm}

El valor esperado de una variable aleatoria tambi\'en se llama media y se representa por
medio del s\'imbolo $\mu$.
\end{frame}

%%%%%%%%%%%%%%%%%%%%%%%%%%%%
%%%%%%%%%%%%%%%%%%%%%%%%%%%%
\begin{frame}
\frametitle{Esperanza, varianza y covarianzas de variables aleatorias}
\framesubtitle{Variables discretas: Varianza y desviaci\'on t\'ipica de una variable aleatoria discreta}

Sea $X$ una variable aleatoria discreta. La esperanza de los cuadrados de las diferencias con
respecto a la media, $(X-\mu)^2$, se llama \textbf{varianza}, se representa por medio del s\'imbolo $\sigma^2$ y viene dada por

\begin{equation*}
  \sigma^2=E[(X-\mu)^2]=\sum_{x}(x-\mu)^2P(x)
\end{equation*}

La varianza de una variable aleatoria discreta X tambi\'en puede expresarse de la forma siguiente:

\begin{equation*}
  \sigma^2=E[X^2]-\mu^2=\sum_{x}x^2P(x)-\mu_x^2
\end{equation*}

La desviaci\'on t\'ipica, $\sigma_x$, es la ra\'iz cuadrada positiva de la varianza.
\end{frame}


%%%%%%%%%%%%%%%%%%%%%%%%%%%%
%%%%%%%%%%%%%%%%%%%%%%%%%%%%
\begin{frame}
\frametitle{Esperanza, varianza y covarianzas de variables aleatorias}
\framesubtitle{Variables discretas: Funci\'on de probabilidad conjunta}

Sean $X$ e $Y$ un par de variables aleatorias discretas. Su funci\'on de probabilidad conjunta
expresa la probabilidad de que simult\'aneamente X tome el valor espec\'ifico $x$ e $Y$ tome el valor
y como funci\'on de $x$ e $y$. La notaci\'on empleada es P(x, y), de donde

\begin{equation*}
  P(x,y)=P(X=x \cap Y=y)
\end{equation*}

\end{frame}

%%%%%%%%%%%%%%%%%%%%%%%%%%%%
%%%%%%%%%%%%%%%%%%%%%%%%%%%%
\begin{frame}
\frametitle{Esperanza, varianza y covarianzas de variables aleatorias}
\framesubtitle{Variables discretas: Obtenci\'on de la funci\'on de probabilidad marginal}

Sean $X$ e $Y$ un par de variables aleatorias distribuidas conjuntamente. En este contexto, la funci\'on
de probabilidad de la variable aleatoria $X$ se llama funci\'on de probabilidad marginal y
se obtiene sumando las probabilidades conjuntas correspondientes a todos los valores posibles;
es decir,

\begin{equation*}
  P(x)=\sum_{y}P(x,y)
\end{equation*}

Asimismo, la funci\'on de probabilidad marginal de la variable aleatoria $Y$ es

\begin{equation*}
  P(y)=\sum_{x}P(x,y)
\end{equation*}

\end{frame}

%%%%%%%%%%%%%%%%%%%%%%%%%%%%
%%%%%%%%%%%%%%%%%%%%%%%%%%%%
\begin{frame}
\frametitle{Esperanza, varianza y covarianzas de variables aleatorias}
\framesubtitle{Variables discretas: Covarianza}

Sea $X$ una variable aleatoria de media $\mu_X$ e $Y$ una variable aleatoria de media $\mu_Y$. El valor esperado
de $(X-\mu_X)(Y-\mu_Y)$ se llama covarianza entre $X$ e $Y$ y se representa por medio de
$Cov(X,Y)$. En el caso de las variables aleatorias discretas,

\begin{equation*}
  Cov(X,Y)=E[(X-\mu_X)(Y-\mu_Y)]=\sum_{x}\sum_{y}(x-\mu_X)(y-\mu_Y)P(x,y)
\end{equation*}

Una expresi\'on equivalente es

\begin{equation*}
  Cov(X,Y)=E[XY]-\mu_X\mu_Y=\sum_{x}\sum_{y}xyP(xy)-\mu_X\mu_Y
\end{equation*}
\end{frame}


%%%%%%%%%%%%%%%%%%%%%%%%%%%%
%%%%%%%%%%%%%%%%%%%%%%%%%%%%
\begin{frame}
\frametitle{Esperanza, varianza y covarianzas de variables aleatorias}
\framesubtitle{Variables discretas: Correlaci\'on}

Sean $X$ e $Y$ variables aleatorias distribuidas conjuntamente. La correlaci\'on entre $X$ e $Y$ es

\begin{equation*}
  \rho=Corr(X,Y)=\frac{Cov(X,Y)}{\sigma_X\sigma_Y}
\end{equation*}

\end{frame}


%%%%%%%%%%%%%%%%%%%%%%%%%%%%
%%%%%%%%%%%%%%%%%%%%%%%%%%%%
\begin{frame}
\frametitle{Esperanza, varianza y covarianzas de variables aleatorias}
\framesubtitle{Valor esperado de las funciones de variables aleatorias}

Sea X una variable aleatoria cuya funci\'on de probabilidad es $P(x)$ y sea $g(X)$ una funci\'on de X.
El valor esperado, $E[g(X)]$, de esa funci\'on se define de la forma siguiente:

\begin{align*}
  E[g(X)]&=\sum_{x}g(x)P(x)\\
  E[g(X)]&=\int_{x}g(x)f(x)dx
\end{align*}

\end{frame}
%%%%%%%%%%%%%%%%%%%%%%%%%%%%
%%%%%%%%%%%%%%%%%%%%%%%%%%%%
\begin{frame}
\frametitle{Esperanza, varianza y covarianzas de variables aleatorias}
\framesubtitle{Variables discretas: Resumen de las propiedades de las funciones lineales
de una variable aleatoria}

Sea X una variable aleatoria de media $\mu_x$ y varianza $\sigma_x^2$ y sean $a$ y $b$ unos n\'umeros fijos constantes cualesquiera. Definamos la variable aleatoria $Y$ como $a+bX$. Entonces, la media y la
varianza de Y son

\begin{align*}
  \mu_Y&=E(a+bX)=a+b\mu_X \\
  \sigma_Y^2&=Var(a+bX)=b^2\sigma_X^2
\end{align*}

por lo que la desviaci\'on t\'ipica de $Y$ es

\begin{align*}
  \sigma_Y &=|b1| \sigma_X
\end{align*}
\end{frame}

%%%%%%%%%%%%%%%%%%%%%%%%%%%%
%%%%%%%%%%%%%%%%%%%%%%%%%%%%
%%%%%%%%%%%%%%%%%%%%%%%%%%%%
%%%%%%%%%%%%%%%%%%%%%%%%%%%%
%%%%%%%%%%%%%%%%%%%%%%%%%%%%
%%%%%%%%%%%%%%%%%%%%%%%%%%%%
\subsubsection{Variables continuas}
\begin{frame}
\frametitle{Esperanza, varianza y covarianzas de variables aleatorias}
\framesubtitle{Variables continuas: Funci\'on de distribuci\'on acumulada}

La \textbf{funci\'on de distribuci\'on acumulada}, $F(x)$, de una variable aleatoria continua $X$ expresa la
probabilidad de que $X$ no sea mayor que el valor de $x$, en funci\'on de $x$

\begin{equation*}
  F(x)= P(X\leq x)
\end{equation*}
\end{frame}

%%%%%%%%%%%%%%%%%%%%%%%%%%%%
%%%%%%%%%%%%%%%%%%%%%%%%%%%%
\begin{frame}
\frametitle{Esperanza, varianza y covarianzas de variables aleatorias}
\framesubtitle{Variables continuas: Probabilidad de un intervalo utilizando una funci\'on
de distribuci\'on acumulada}

Sea $X$ una variable aleatoria continua que tiene una funci\'on de distribuci\'on acumulada $F(x)$ y
sean $a$ y $b$ dos valores posibles de $X$, siendo $a<b$. La probabilidad de que $X$ se encuentre
entre $a$ y $b$ es

\begin{equation*}
  P(a<X<b)=F(b)-F(a)
\end{equation*}
\end{frame}

%%%%%%%%%%%%%%%%%%%%%%%%%%%%
%%%%%%%%%%%%%%%%%%%%%%%%%%%%
\begin{frame}
\frametitle{Esperanza, varianza y covarianzas de variables aleatorias}
\framesubtitle{Variables continuas: Funci\'on de densidad de probabilidada}

Sea $X$ una variable aleatoria continua y $x$ cualquier n\'umero situado en el rango de valores que
puede tomar esta variable aleatoria. La funci\'on de densidad de probabilidad, $f(x)$, de la variable
aleatoria es una funci\'on que tiene las siguientes propiedades:

\begin{itemize}
  \item [1.] $f(x)>0$ para todos los valores de x.
  \item [2.] El \'area situada debajo de la funci\'on de densidad de probabilidad, $f(x)$, cuando se abarcan
todos los valores de la variable aleatoria, $X$, es igual a 1,0.
  \item [3.] Supongamos que se representa gr\'aficamente esta funci\'on de densidad. Sean $a$ y $b$ dos
valores posibles de la variable aleatoria $X$, siendo $a<b$. En ese caso, la probabilidad
de que $X$ se encuentre entre $a$ y $b$ es el \'area situada debajo de la funci\'on de densidad
entre estos puntos.
\end{itemize}
\end{frame}

%%%%%%%%%%%%%%%%%%%%%%%%%%%%
%%%%%%%%%%%%%%%%%%%%%%%%%%%%
\begin{frame}
\frametitle{Esperanza, varianza y covarianzas de variables aleatorias}
\framesubtitle{Variables continuas: Funci\'on de densidad de probabilidada}

\begin{itemize}
  \item [4.] La funci\'on de distribuci\'on acumulada, $F(x_0)$, es el \'area situada debajo de la funci\'on de
densidad de probabilidad, $f(x)$, hasta $x_0$:

\begin{equation*}
  F(x_0)=\int_{x_m}^{x_0}f(x)dx
\end{equation*}
  \item [] donde $x_m$ es el valor m\'inimo de la variable aleatoria $X$.
\end{itemize}
\end{frame}


\subsection{Esperanza y varianza de combinaciones lineales de variables aleatorias}

%%%%%%%%%%%%%%%%%%%%%%%%%%%%
%%%%%%%%%%%%%%%%%%%%%%%%%%%%
\begin{frame}
\frametitle{Esperanza y varianza de combinaciones lineales de variables aleatorias}
\framesubtitle{Sumas de variables aleatorias}

Sean $X_1,X_2,\dots,X_K$, $K$ variables aleatorias que tienen las medias \\
$\mu_1,\mu_2,\dots,\mu_K$ y las varianzas
$\sigma_1^2,\sigma_2^2,\dots,\sigma_k^2$. Se cumplen las siguientes propiedades:

\begin{itemize}
  \item [1.] La media de su suma es la suma de sus medias; es decir,
  \begin{equation*}
    E(X_1+X_2+\dots+X_K)=\mu_1+\mu_2+\dots+\mu_K
  \end{equation*}
  \item [2.] Si la covarianza entre cada par de estas variables aleatorias es $0$, entonces la varianza
de su suma es la suma de sus varianzas; es decir,
    \begin{equation*}
    Var(X_1+X_2+\dots+X_K)=\sigma_1^2+\sigma_2^2+\dots+\sigma_K^2
  \end{equation*}
  \item [] Sin embargo, si las covarianzas entre pares de variables aleatorias no son 0, la varianza
de su suma es
    \begin{equation*}
    Var(X_1+X_2+\dots+X_K)=\sigma_1^2+\sigma_2^2+\dots+\sigma_K^2+2\sum_{i=1}^{K-1}\sum_{j=i+1}^{K}Cov(X_i,X_j)
  \end{equation*}
\end{itemize}

\end{frame}

%%%%%%%%%%%%%%%%%%%%%%%%%%%%
%%%%%%%%%%%%%%%%%%%%%%%%%%%%
\begin{frame}
\frametitle{Esperanza y varianza de combinaciones lineales de variables aleatorias}
\framesubtitle{Diferencias entre un par de variables aleatorias}

Sean $X$ e $Y$ un par de variables aleatorias que tienen las medias $\mu_X$ y $\mu_Y$ y las varianzas $\sigma_X^2$ y $\sigma_Y^2$. Se cumplen las siguientes propiedades:


\begin{itemize}
  \item [1.] La media de su diferencia es la diferencia de sus medias; es decir,
  \begin{equation*}
    E(X-Y)=\mu_X-\mu_Y
  \end{equation*}
  \item [2.] Si la covarianza entre $X$ e $Y$ es 0, entonces la varianza de su diferencia es
    \begin{equation*}
    Var(X-Y)=\sigma_X^2+\sigma_Y^2
  \end{equation*}
  \item [3.] Si la covarianza entre $X$ e $Y$ no es 0, entonces la varianza de su diferencia es
    \begin{equation*}
    Var(X-Y)=\sigma_X^2+\sigma_Y^2-Cov(X,Y)
  \end{equation*}
\end{itemize}

\end{frame}
%%%%%%%%%%%%%%%%%%%%%%%%%%%%
%%%%%%%%%%%%%%%%%%%%%%%%%%%%
\begin{frame}
\frametitle{Esperanza y varianza de combinaciones lineales de variables aleatorias}
\framesubtitle{Variables discretas: Resumen de las propiedades de las funciones lineales
de una variable aleatoria}

\begin{enumerate}
    \item La variable aleatoria continua X tiene una funci\'on densidad de probabilidad dada por $f(x) = cx, 0<x<1$, siendo c una constante positiva. Determine Var(X).
    \item Sean X e Y dos variable aleatorias con primeros y segundos momentos dados por $E(X)=\mu_X, E(y)=\mu_Y, Var(X) = \sigma_{XX}, Var(X) = \sigma_{YY}$, y $cov(X,Y) = \sigma_{XY}$. Def\'inase $\gamma=\sigma_{XY}/\sigma_{XX}$ y $\alpha=\mu_Y - \gamma \mu_X$.
    \begin{enumerate}
        \item Demuestre que la variable aleatoria z= ($Y-\alpha -\gamma X$) no est\'a correlacionada con X. Encuentre su media $E[z]$ y varianza $Var(z)$.
        \item Para constantes arbitrarias a y b, considere la variable aleatoria $Z=Y-a-bX$. Encuentre los valores de a y b que minimizan $E(Z^2)$.
    \end{enumerate}
    \item Si $Var(X_1) = 5$, $Var(X_2) = 4$, $Var(X_3)=7$, $Cov(X_1, X_2) = 3$, $Cov(X_1,X_3)= -2$ y $X_2$ y $X_3$ son independientes, encuentre la covarianza de $Y_1=X_1-2X_2+3X_3$ e $Y_2 = -2X_1+3X_2+4X_3$
\end{enumerate}

\end{frame}



\end{document} 